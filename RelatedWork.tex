\section{Related Work}
\label{sec:relatedwork}
%Recent works that evaluate alternative transport protocols for adaptive bitrate (ABR) video streaming include the work by McQuistin et al. \cite{McQuistin}. The authors present \textit{TCP Hollywood}, a TCP variant, which implements out-of-order delivery and inconsistent retransmissions in order to improve good-put of video streaming applications. 
A recent paper by Google \cite{Langley:SIGCOMM:2017} provides a detailed analysis of an Internet-scale deployment of QUIC. They specifically look at latency and rebuffer rate in order to understand the performance implications of QUIC for video streaming over YouTube. Timmerer et al. \cite{timmerer2016advanced} evaluate ABR streaming over QUIC for varying network latencies and show that there is no significant benefit to QoE streaming with the use of QUIC. In \cite{szaboquic}, a demonstration by Szab\'o et al. provides a new congestion control mechanism for QUIC that aggressively varies download rate according to a buffer-based priority level assigned by the ABR streaming client.
Carlucci et al. \cite{Carlucci:2015} present results that compare TCP and QUIC under varying network conditions and buffer size. In~\cite{Kakhki:IMC:2017}, Kakhki et al. perform a detailed analysis of QUIC under varying network conditions to investigate the benefits of using QUIC for applications such as web browsing and video streaming over YouTube. The authors of \cite{AYAD201890} also compare the performance of several rate adaptive DASH players including QUIC and conclude that QUIC is more aggressive compared to TCP. The authors of \cite{BHayes} devise and deploy an SDN approach to to improve the QoE of ABR streaming by monitoring MPTCP retransmissions where their system dynamically switches between network paths and protocols to mitigate re-ordering effects. While we similarly compare the performance of TCP (using HTTP/1.1 and HTTP/2) with QUIC, our work is more focussed on the potential benefits that QUIC can provide for video streaming especially with respect to retransmitting video segments in higher qualities. Similar experiments are performed by the authors of \cite{huysegems2015http}, where they use the multiplexing feature of HTTP/2 to simultaneously request multiple qualities of a segment. While retransmissions can be regarded as an additional burden on the available bandwidth we note that recent works such as~\cite{Vulimiri:CoNext} suggest different types of redundant transmission to provide higher QoS. In contrast to~\cite{huysegems2015http} and~\cite{Vulimiri:CoNext}, we only invoke retransmissions in a systematic way, thereby guaranteeing an improvement in QoE while also minimizing the consumption of additional bandwidth. Moreover, in order to analyze the implications of specific network conditions that affect ABR video streaming, we design, develop and prototype such a system in a nearly isolated, controlled testbed environment.

Legacy protocols that perform adaptive bitrate video streaming over UDP include systems such as Real-time Transport Protocol (RTP) \cite{rtp} and Stream Control Transport Protocol (SCTP) \cite{sctp}. Similar to QUIC, SCTP also allows multiplexing of multiple chunks into one packet and avoids HOL blocking, thus, allowing unordered delivery to the application layer. Unlike QUIC, SCTP implements 
congestion control according to the TCP \textit{NewReno} specification which uses Selective Acknowledgement (SACK) for loss recovery.
% the \textit{NewReno} form of TCP congestion control including TCP's method of Selective Acknowledgement (SACK) for loss recovery. 
Another example of an ABR protocol over UDP is the Video Transport Protocol (VTP) which was designed and evaluated by Balk et al. \cite{balk2003adaptive}. In this work, the authors employ a form of congestion avoidance where the sending rate at the server is increased by a single packet for every RTT measurement. This design is different from the AIMD congestion control employed by TCP and QUIC since it eliminates the effect of slow start and attempts to provide an accurate estimate of the available bandwidth in the network. Some drawbacks of this approach are the requirement of two UDP sockets for every connection and the use of Berkeley Packet Filters to collect timestamps at the server and client for every video stream, thus, reducing both performance and scalability of the system. Although there are a number of server push approaches such as \cite{huysegems2015http} and \cite{xiao2016evaluating} that have been proposed for HTTP/2, adapting such systems for retransmissions would not scale since the computation and storage overhead incurred on the server per individual client connection would render such an approach infeasible.
\begin{comment}
MORE RELATED WORK:

HotNets 2017~\cite{Narayan:HotNets:2017} \DB{This is only remotely related as it looks at software defined congestion control}

Sigcomm 2017 Carousel~\cite{Saeed:Sigcomm:2017} \DB{Remotely related as they look at timing wheels and rate control}
\end{comment}
