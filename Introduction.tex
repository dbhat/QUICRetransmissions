\section{Introduction}
After two decades, the HyperText Transfer Protocol has undergone a significant makeover resulting in the introduction of the HTTP/2 standard~\cite{Stenberg:CCR:2014} gaining notable popularity in the Internet, where it is currently used by 24.6\% of all web sites ~\cite{http2_pop}. HTTP/2 makes several improvements over its predecessor HTTP/1.1~\cite{Fielding:RFC:1999}. These improvements include {\it a)} multiplexing, where streams for multiple requests can be sent over a single TCP session; {\it b)} header compression; and, {\it c)} an option where the web server can push content to the client proactively. HTTP/2 has been specified to use TCP as the underlying transport protocol. This combination of HTTP/2 and TCP has several performance issues, including a delay introduced by the 3-way handshake for each connection setup (this is even higher if Transport Layer Security (TLS) is used). In addition, the issue of head of line (HOL) blocking still exists. The Quick UDP Internet Connections protocol (QUIC)~\cite{Langley:SIGCOMM:2017} is a new approach designed to combine the speed of UDP with the reliability of TCP and, thus overcome these issues. QUIC has been specifically designed to reduce latency of web page loads and mitigate rebuffers in video streaming clients.

Adaptive bitrate (ABR) streaming has become the de-facto streaming standard for video on demand platforms such as Netflix \cite{netflix} and Youtube \cite{youtube}. With more than 70\% of the peak hour US Internet traffic~\cite{sandvine-16}, video streaming has become \emph{the} killer-application of today's Internet. ABR video streaming solutions like Dynamic Adaptive Streaming over HTTP (DASH)~\cite{DASH2011} are, however, stuck in an HTTP/TCP setting that has been shown to possess substantial drawbacks with respect to Quality-of-Experience (QoE)~\cite{confused-timid, WangRZ16}.

Recently, we proposed a DASH-based ABR approach (SQUAD)~\cite{WangRZ16} that has the goal to improve QoE for viewers watching video streams over the Internet. One specific feature of SQUAD is the ability to retransmit segments\footnote{In the remainder of this paper, we use the word segments to denote ABR video segments unless specified otherwise.} in a higher quality than they were originally transmitted in~\cite{Wang:TOMM:2017} to reduce frequent quality changes during a streaming session. 
%In this paper, we first present results from an analysis of a large data set of streaming sessions, which reveals that $\sim$50\% of the sessions experience such quality changes.
The drawback of implementing this approach on top of HTTP/1.1 is the inability to efficiently schedule such retransmissions. In the case of one TCP session, retransmission requests\footnote{In the remainder of this paper, we use the word retransmissions to denote retransmissions of a received video segment in a higher quality unless specified otherwise.} have to be interleaved with requests for new original segments, that have not been requested in the past. Parallel transmissions require the setup of a new connection, which comes with the drawback of additional delay due to the 3-way handshake.
The use of HTTP/2 over TCP makes such retransmissions more efficient, since they can be scheduled within the same TCP connection. 
%\sout{and have already been observed to reduce page load times where multiple objects embedded in a web page are downloaded simultaneously~\cite{}}. 
While HTTP/2 has the potential to improve the performance of SQUAD in the case of retransmission, the impact of losses and the resulting HOL blocking has not been studied. 
In addition, it has not been evaluated to what extent QUIC can further improve SQUAD with retransmissions, since it eliminates the HOL blocking issue. 
% While it is known that QUIC resolves the HOL blocking issue, it is unclear to what extent QUIC can further improve SQUAD with retransmissions. - Use one of the two sentences}

In this paper, we make the following contributions:
\vspace{-5pt}
\begin{itemize}
\item An analysis of 5 million video streaming sessions reveals that switches in quality representations that result in a gap occur in almost 36\% of all streaming sessions. In the case of mobile clients this number increases to 50\%.
\item We make a systematic comparison of the multiplexing feature of HTTP/2 and QUIC, particularly for retransmitting ABR video segments in a higher quality with the objective of improving the overall QoE of ABR streaming sessions.
\item Our evaluation results show that QUIC retransmissions can \textit{significantly} improve the average quality bitrate while simultaneously \textit{minimizing} bit rate variations over the duration of a streaming session.
%\item \DB{Retransmit segments using SQUAD and test with HTTP/2 and QUIC}
%\item \DB{Investigate the effect of reordering segments on ABR streaming with SQUAD}
%\item \DB{Measurements in the wild over inter and intra continental links}
\end{itemize}


The remainder of this paper is structured as follows. In Sect.~\ref{sec:background}, we present background information on HTTP/2 and QUIC, as well as retransmission scheduling, and an analysis of more then 5 million streaming sessions from a CDN. Sect.~\ref{sec:retrans} details how segment retransmission is performed in the case of HTTP1.1, HTTP/2, and QUIC and an evaluation of this approach using controlled and Internet measurements is presented in Sect.~\ref{sec:eval}. Related work is presented in Sect.~\ref{sec:relatedwork} and Sect.~\ref{sec:conc} concludes the paper.